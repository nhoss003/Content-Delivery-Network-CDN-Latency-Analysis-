%% LyX 2.1.4 created this file.  For more info, see http://www.lyx.org/.
%% Do not edit unless you really know what you are doing.
\documentclass{beamer}
\usepackage{mathptmx}
\usepackage{graphicx}
\usepackage{listings}
\usepackage{minted}
\usepackage{xcolor}

\definecolor{LightGray}{gray}{0.9}

\ifx\hypersetup\undefined
  \AtBeginDocument{%
    \hypersetup{unicode=true,
 bookmarksnumbered=false,bookmarksopen=false,
 breaklinks=false,pdfborder={0 0 0},colorlinks=false}
  }
\else
  \hypersetup{unicode=true,
 bookmarksnumbered=false,bookmarksopen=false,
 breaklinks=false,pdfborder={0 0 0},colorlinks=false}
\fi

\makeatletter
%%%%%%%%%%%%%%%%%%%%%%%%%%%%%% Textclass specific LaTeX commands.
 % this default might be overridden by plain title style
 \newcommand\makebeamertitle{\frame{\maketitle}}%
 % (ERT) argument for the TOC
 \AtBeginDocument{%
   \let\origtableofcontents=\tableofcontents
   \def\tableofcontents{\@ifnextchar[{\origtableofcontents}{\gobbletableofcontents}}
   \def\gobbletableofcontents#1{\origtableofcontents}
 }

%%%%%%%%%%%%%%%%%%%%%%%%%%%%%% User specified LaTeX commands.
\usetheme{Warsaw}
% or ...
\useoutertheme{infolines}
\addtobeamertemplate{headline}{}{\vskip2pt}

\setbeamercovered{transparent}
% or whatever (possibly just delete it)



\setbeamertemplate{footline}
{
  \leavevmode%
  \hbox{%
  \begin{beamercolorbox}[wd=.5\paperwidth,ht=2.25ex,dp=1ex,center]{title in head/foot}%
    \usebeamerfont{title in head/foot}\insertshorttitle
  \end{beamercolorbox}%
  \begin{beamercolorbox}[wd=.5\paperwidth,ht=2.25ex,dp=1ex,right]{date in head/foot}%
    \usebeamerfont{date in head/foot}\insertshortdate{}\hspace*{2em}
    \insertframenumber{} / \inserttotalframenumber\hspace*{2ex} 
  \end{beamercolorbox}}%
  \vskip0pt%
}

\usepackage{pifont}
\makeatother

\begin{document}

\title{CDN Latency}
\subtitle{CDN Latency}
\author{Chris, Nilou and Andres}
\institute{University of California, Riverside}

\makebeamertitle

% \AtBeginSection[]{
%   \frame<beamer>{ 
%     \frametitle{Agenda}   
%     \tableofcontents[currentsubsection] 
%   }
% }

\newif\iflattersubsect

\AtBeginSection[] {
    \begin{frame}<beamer>
    \frametitle{Outline} %
    \tableofcontents[currentsection]  
    \end{frame}
    \lattersubsectfalse
}

\AtBeginSubsection[] {
    % \iflattersubsect
    \begin{frame}<beamer>
    \frametitle{Outline} %
    \tableofcontents[currentsubsection]  
    \end{frame}
    % \fi
    % \lattersubsecttrue
}

\begin{frame}{Outline}
  \tableofcontents{}
\end{frame}

\section{Introduction}

\begin{frame}{Introduction}
  \begin{itemize}
    \item Measuring and Evaluating Large-Scale CDNs (Huang et al., 2008).
    \item The goal of a CDN is to serve content to end-users with high availability and high performance. 
  \end{itemize}
\end{frame}

\begin{frame}{Introduction}
  \centering
  \includegraphics[width=0.85\textwidth]{CDN.png}
\end{frame}

\begin{frame}{Introduction}
  \begin{itemize}
    \item But... Where do we deploy a new CDN server???
  \end{itemize}
\end{frame}

\section{Methodology}

\begin{frame}{Overview}
  \begin{enumerate}
    \item Collect information about current CDNs
    \item Select a group of possible candidates
    \item Measure the latency between points (Ping time)
    \item Graph analysis
    \item Visualization
  \end{enumerate}
\end{frame}

\subsection{CDNs collection}

\begin{frame}{Virtual machines}
  \begin{itemize}
    \item Microsoft Azure: Cloud Computing Platform \& Services \footnote{\url{https://azure.microsoft.com/}}
    \item 8 instances in different geographical regions
    \begin{enumerate}
     \item East USA
     \item Central USA
     \item South Central USA
     \item West USA
     \item North Central USA
     \item East Canada
     \item Central Canada
     \item Riverside
    \end{enumerate}
  \end{itemize}
\end{frame}

\begin{frame}{Virtual machines}
  \centering
  \includegraphics[width=0.9\textwidth]{cdns.png}
\end{frame}

\subsection{Choosing candidates}

\begin{frame}{Choosing candidates}
  \begin{itemize}
    \item Universities Worldwide \footnote{\url{http://univ.cc/}}
    \item Universities from USA and Canada
  \end{itemize}
\end{frame}

\begin{frame}{Choosing candidates}
  \begin{itemize}
    \item Cleaning and processing 
    \begin{itemize}
     \item Remove urls which did not respond to ping
     \item Remove duplicates (in the same city)
     \item Remove incorrect format
     \item Extract latitude and longitude (reverse geocoding)
    \end{itemize}
    \item End up with 97 candidates 
  \end{itemize}
\end{frame}

\begin{frame}[fragile]{Python scripts}
  \inputminted[
    fontsize=\tiny,
    tabsize=2,
    breaklines,
    linenos,
    framesep=10pt,
    frame=single
    ]{python}{../getUrls.py}
\end{frame}

\begin{frame}[fragile]{Python scripts}
  \inputminted[
    fontsize=\tiny,
    tabsize=2,
    breaklines,
    linenos,
    framesep=10pt,
    frame=single
    ]{python}{../getUrlInfo.py}
\end{frame}

\begin{frame}{Choosing candidates}
  \centering
  \includegraphics[width=0.9\textwidth]{candidates.png}
\end{frame}

\subsection{Collecting ping data}

\begin{frame}{Ping data}
  \begin{itemize}
    \item We collect ping data from each current CDN to all the candidates
    \item Run a script at each virtual machine
    \item Transmit 5 packages and extract the average ping time
  \end{itemize}
\end{frame}

\begin{frame}[fragile]{Python scripts}
  \inputminted[
    fontsize=\tiny,
    tabsize=2,
    breaklines,
    linenos,
    framesep=10pt,
    frame=single
    ]{python}{../getPingInfo.py}
\end{frame}

\subsection{Building the graph}

\begin{frame}{Building the graph}
  \begin{itemize}
    \item Merge results for each virtual machine (CDN)
    \item Use \texttt{networkX} package in python
    \item Final graph has 105 nodes and 776 edges
  \end{itemize}
\end{frame}

\begin{frame}[fragile]{Python scripts}
  \inputminted[
    fontsize=\tiny,
    tabsize=2,
    breaklines,
    linenos,
    framesep=10pt,
    frame=single
    ]{python}{createGraph.py}
\end{frame}

\begin{frame}{Building the graph}
  \centering
  \includegraphics[width=0.9\textwidth]{graph.png}
\end{frame}

\subsection{Visualization}

\begin{frame}{Visualization}
  \begin{itemize}
    \item Use networkX to save the output to a shapefile format 
    \item Use QGIS to visualize data
  \end{itemize}
\end{frame}

\begin{frame}{Visualization}
  \centering
  \includegraphics[width=0.9\textwidth]{vis.png}
\end{frame}

\section{Results}

\begin{frame}{Results}
  \begin{itemize}
    \item Iterate at each candidates to all the CDNs 
    \item Compute the average ping time
    \item Sort by average time ascendantly
  \end{itemize}
\end{frame}

\begin{frame}{Results}
  \centering
  \includegraphics[width=0.9\textwidth]{results.png}
\end{frame}

\begin{frame}{Results}
  \centering
  \includegraphics[width=0.9\textwidth]{Clarksville.png}
\end{frame}

\begin{frame}{Results}
  \centering
  \includegraphics[width=0.9\textwidth]{bloomsburg.png}
\end{frame}

\subsection*{Thanks...}

\begin{frame}{}
  \centering
  \huge Thank you!!! 
\end{frame}

\end{document}